%%%%%%%%%%%%%%%%%%%%%%%%%%%%%%%%%%%%%%%%%%%%%%%%%%%%%%%%%%%%%%%%%%%%%%%%%%%%%%%%%%%%
% Do not alter this block (unless you're familiar with LaTeX
\documentclass{article}
\usepackage[margin=1in]{geometry} 
\usepackage{amsmath,amsthm,amssymb,amsfonts, fancyhdr, color, comment, graphicx, environ, caption}
\graphicspath{ {./Figures/} }
\usepackage{xcolor}
\usepackage{mdframed}
\usepackage[shortlabels]{enumitem}
\usepackage{indentfirst}
\usepackage{hyperref}
\hypersetup{
	colorlinks=true,
	linkcolor=blue,
	filecolor=magenta,      
	urlcolor=blue,
}


\pagestyle{fancy}


\newenvironment{problem}[2][Problem]
{ \begin{mdframed}[backgroundcolor=gray!20] \textbf{#1 #2} \\}
	{  \end{mdframed}}

% Define solution environment
\newenvironment{solution}
{\textit{Proof:}}
{}

\renewcommand{\qed}{\quad\qedsymbol}

% prevent line break in inline mode
\binoppenalty=\maxdimen
\relpenalty=\maxdimen

%%%%%%%%%%%%%%%%%%%%%%%%%%%%%%%%%%%%%%%%%%%%%
%Fill in the appropriate information below
\lhead{M. Farouk HASNAOUI}
\rhead{Analyse numérique} 
\chead{\textbf{Méthode des différences finies}}
%%%%%%%%%%%%%%%%%%%%%%%%%%%%%%%%%%%%%%%%%%%%%

\begin{document}
	
	\begin{mdframed}[backgroundcolor=blue!20]
		This exercises is writing by Prf. Kamel MEFTAH, University of Biskra, and M.Farouk HASNAOUI National Polytechnic School of Algiers.
	\end{mdframed}

\begin{problem}{1}
	\textcolor{blue}{Différences finies pour les ODEs} \\
	Considérons le problème aux limites suivant :
	\begin{equation*}
		y^{''}(x) - \Bigr(1-\frac{x}{5}\Bigr)y(x) = x
		\end{equation*}
	avec $y(1) = 2$ et $y(3) = 1$ \\
	
	$\bullet$ Trouver les valeurs approximation de y aux points : $x_1$ = 1.5, $x_2$ = 2 et $x_3$ = 2.5 
	\end{problem}

\begin{problem}{2}
	\textcolor{blue}{Classification des EDPs} \\
	Trouver le type de ces EDPs :
	\begin{eqnarray*}
		\frac{\partial^2u}{\partial x^2} + \frac{\partial^2u}{\partial y^2} &=& 0 \\
		\frac{\partial^2u}{\partial x^2} &=& \frac{1}{c^2} \frac{\partial^2u}{\partial t^2} \\
		\frac{\partial^2u}{\partial x^2} &=& \frac{1}{\alpha}\frac{\partial u}{\partial t} \\
		\frac{\partial^2\psi}{\partial x^2} + \frac{\partial^2\psi}{\partial y^2} &=& -4\pi\rho \\
		\frac{\partial u}{\partial t} + \alpha\frac{\partial u}{\partial x} &=& 0 \\
		\frac{\partial^2\psi}{\partial x^2} &=& \frac{1}{c^2}\frac{\partial^2\psi}{\partial t^2} - \frac{\rho}{\epsilon} \\
		\frac{\partial u}{\partial x} + \frac{\partial v}{\partial y} &=& 0
		\end{eqnarray*}
\end{problem}

\begin{problem}{3}
	\textcolor{blue}{Résolution des EDPs éliptiques (Cas simple)} \\
	Soit l'équation de Laplace suivante :
	\begin{equation*}
		\frac{\partial^2u}{\partial x^2} + \frac{\partial^2u}{\partial y^2} = 0
	\end{equation*}
	avec le maillage (Fig. \ref{maillage_Laplace})
	
	$\bullet$ Résoudre cette EDP utilisant la méthode des différences finies.
\end{problem}
\begin{figure}[!h]
	\centering
	\includegraphics[width=.45\textwidth, height=.3\textheight]{maillage_Laplace.png}
	\caption{Le maillage de la problème 3.}\label{maillage_Laplace}
	\end{figure}
\newpage

\begin{problem}{4}
	\textcolor{blue}{Résolution des EDPs éliptiques} \\
	Soit l'équation de Poisson en 2D (x, y) suivante :
	\begin{equation*}
		\Delta T = \sin{x} + T
		\end{equation*}
	avec les conditions aux limites suivantes :
	\begin{eqnarray*}
		\varphi(y = 0) &=& 2 \\
		\varphi(x = 0) &=& \varphi_{\text{convection}} \\
		T(x = L) &=& \sin{(\pi y^2)} 
		\end{eqnarray*}
	
	$\bullet$ Résoudre cette EDP utilisant la méthode des différences finies, si : \\
	Il y'a une symétrie á : y = H = $\frac{1}{2}$, L = 1, h = $\Delta x$ = $\frac{1}{2}$, et h = $\Delta y$ = $\frac{1}{4}$.
\end{problem}

\begin{problem}{5}
	\textcolor{blue}{Résolution des EDPs paraboliques} \\
	Résoudre l'EDP suivante :
	\begin{equation*}
		u_t = u_{xx}
	\end{equation*}
avec : $0 \leq x \leq 1$, $u(x,0) = \sin{(\pi x)}$, h=1/3, k=1/18, et $\Delta t = 1/6$
\end{problem}

\end{document}